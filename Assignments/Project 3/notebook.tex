
% Default to the notebook output style

    


% Inherit from the specified cell style.




    
\documentclass[11pt]{article}

    
    
    \usepackage[T1]{fontenc}
    % Nicer default font (+ math font) than Computer Modern for most use cases
    \usepackage{mathpazo}

    % Basic figure setup, for now with no caption control since it's done
    % automatically by Pandoc (which extracts ![](path) syntax from Markdown).
    \usepackage{graphicx}
    % We will generate all images so they have a width \maxwidth. This means
    % that they will get their normal width if they fit onto the page, but
    % are scaled down if they would overflow the margins.
    \makeatletter
    \def\maxwidth{\ifdim\Gin@nat@width>\linewidth\linewidth
    \else\Gin@nat@width\fi}
    \makeatother
    \let\Oldincludegraphics\includegraphics
    % Set max figure width to be 80% of text width, for now hardcoded.
    \renewcommand{\includegraphics}[1]{\Oldincludegraphics[width=.8\maxwidth]{#1}}
    % Ensure that by default, figures have no caption (until we provide a
    % proper Figure object with a Caption API and a way to capture that
    % in the conversion process - todo).
    \usepackage{caption}
    \DeclareCaptionLabelFormat{nolabel}{}
    \captionsetup{labelformat=nolabel}

    \usepackage{adjustbox} % Used to constrain images to a maximum size 
    \usepackage{xcolor} % Allow colors to be defined
    \usepackage{enumerate} % Needed for markdown enumerations to work
    \usepackage{geometry} % Used to adjust the document margins
    \usepackage{amsmath} % Equations
    \usepackage{amssymb} % Equations
    \usepackage{textcomp} % defines textquotesingle
    % Hack from http://tex.stackexchange.com/a/47451/13684:
    \AtBeginDocument{%
        \def\PYZsq{\textquotesingle}% Upright quotes in Pygmentized code
    }
    \usepackage{upquote} % Upright quotes for verbatim code
    \usepackage{eurosym} % defines \euro
    \usepackage[mathletters]{ucs} % Extended unicode (utf-8) support
    \usepackage[utf8x]{inputenc} % Allow utf-8 characters in the tex document
    \usepackage{fancyvrb} % verbatim replacement that allows latex
    \usepackage{grffile} % extends the file name processing of package graphics 
                         % to support a larger range 
    % The hyperref package gives us a pdf with properly built
    % internal navigation ('pdf bookmarks' for the table of contents,
    % internal cross-reference links, web links for URLs, etc.)
    \usepackage{hyperref}
    \usepackage{longtable} % longtable support required by pandoc >1.10
    \usepackage{booktabs}  % table support for pandoc > 1.12.2
    \usepackage[inline]{enumitem} % IRkernel/repr support (it uses the enumerate* environment)
    \usepackage[normalem]{ulem} % ulem is needed to support strikethroughs (\sout)
                                % normalem makes italics be italics, not underlines
    

    
    
    % Colors for the hyperref package
    \definecolor{urlcolor}{rgb}{0,.145,.698}
    \definecolor{linkcolor}{rgb}{.71,0.21,0.01}
    \definecolor{citecolor}{rgb}{.12,.54,.11}

    % ANSI colors
    \definecolor{ansi-black}{HTML}{3E424D}
    \definecolor{ansi-black-intense}{HTML}{282C36}
    \definecolor{ansi-red}{HTML}{E75C58}
    \definecolor{ansi-red-intense}{HTML}{B22B31}
    \definecolor{ansi-green}{HTML}{00A250}
    \definecolor{ansi-green-intense}{HTML}{007427}
    \definecolor{ansi-yellow}{HTML}{DDB62B}
    \definecolor{ansi-yellow-intense}{HTML}{B27D12}
    \definecolor{ansi-blue}{HTML}{208FFB}
    \definecolor{ansi-blue-intense}{HTML}{0065CA}
    \definecolor{ansi-magenta}{HTML}{D160C4}
    \definecolor{ansi-magenta-intense}{HTML}{A03196}
    \definecolor{ansi-cyan}{HTML}{60C6C8}
    \definecolor{ansi-cyan-intense}{HTML}{258F8F}
    \definecolor{ansi-white}{HTML}{C5C1B4}
    \definecolor{ansi-white-intense}{HTML}{A1A6B2}

    % commands and environments needed by pandoc snippets
    % extracted from the output of `pandoc -s`
    \providecommand{\tightlist}{%
      \setlength{\itemsep}{0pt}\setlength{\parskip}{0pt}}
    \DefineVerbatimEnvironment{Highlighting}{Verbatim}{commandchars=\\\{\}}
    % Add ',fontsize=\small' for more characters per line
    \newenvironment{Shaded}{}{}
    \newcommand{\KeywordTok}[1]{\textcolor[rgb]{0.00,0.44,0.13}{\textbf{{#1}}}}
    \newcommand{\DataTypeTok}[1]{\textcolor[rgb]{0.56,0.13,0.00}{{#1}}}
    \newcommand{\DecValTok}[1]{\textcolor[rgb]{0.25,0.63,0.44}{{#1}}}
    \newcommand{\BaseNTok}[1]{\textcolor[rgb]{0.25,0.63,0.44}{{#1}}}
    \newcommand{\FloatTok}[1]{\textcolor[rgb]{0.25,0.63,0.44}{{#1}}}
    \newcommand{\CharTok}[1]{\textcolor[rgb]{0.25,0.44,0.63}{{#1}}}
    \newcommand{\StringTok}[1]{\textcolor[rgb]{0.25,0.44,0.63}{{#1}}}
    \newcommand{\CommentTok}[1]{\textcolor[rgb]{0.38,0.63,0.69}{\textit{{#1}}}}
    \newcommand{\OtherTok}[1]{\textcolor[rgb]{0.00,0.44,0.13}{{#1}}}
    \newcommand{\AlertTok}[1]{\textcolor[rgb]{1.00,0.00,0.00}{\textbf{{#1}}}}
    \newcommand{\FunctionTok}[1]{\textcolor[rgb]{0.02,0.16,0.49}{{#1}}}
    \newcommand{\RegionMarkerTok}[1]{{#1}}
    \newcommand{\ErrorTok}[1]{\textcolor[rgb]{1.00,0.00,0.00}{\textbf{{#1}}}}
    \newcommand{\NormalTok}[1]{{#1}}
    
    % Additional commands for more recent versions of Pandoc
    \newcommand{\ConstantTok}[1]{\textcolor[rgb]{0.53,0.00,0.00}{{#1}}}
    \newcommand{\SpecialCharTok}[1]{\textcolor[rgb]{0.25,0.44,0.63}{{#1}}}
    \newcommand{\VerbatimStringTok}[1]{\textcolor[rgb]{0.25,0.44,0.63}{{#1}}}
    \newcommand{\SpecialStringTok}[1]{\textcolor[rgb]{0.73,0.40,0.53}{{#1}}}
    \newcommand{\ImportTok}[1]{{#1}}
    \newcommand{\DocumentationTok}[1]{\textcolor[rgb]{0.73,0.13,0.13}{\textit{{#1}}}}
    \newcommand{\AnnotationTok}[1]{\textcolor[rgb]{0.38,0.63,0.69}{\textbf{\textit{{#1}}}}}
    \newcommand{\CommentVarTok}[1]{\textcolor[rgb]{0.38,0.63,0.69}{\textbf{\textit{{#1}}}}}
    \newcommand{\VariableTok}[1]{\textcolor[rgb]{0.10,0.09,0.49}{{#1}}}
    \newcommand{\ControlFlowTok}[1]{\textcolor[rgb]{0.00,0.44,0.13}{\textbf{{#1}}}}
    \newcommand{\OperatorTok}[1]{\textcolor[rgb]{0.40,0.40,0.40}{{#1}}}
    \newcommand{\BuiltInTok}[1]{{#1}}
    \newcommand{\ExtensionTok}[1]{{#1}}
    \newcommand{\PreprocessorTok}[1]{\textcolor[rgb]{0.74,0.48,0.00}{{#1}}}
    \newcommand{\AttributeTok}[1]{\textcolor[rgb]{0.49,0.56,0.16}{{#1}}}
    \newcommand{\InformationTok}[1]{\textcolor[rgb]{0.38,0.63,0.69}{\textbf{\textit{{#1}}}}}
    \newcommand{\WarningTok}[1]{\textcolor[rgb]{0.38,0.63,0.69}{\textbf{\textit{{#1}}}}}
    
    
    % Define a nice break command that doesn't care if a line doesn't already
    % exist.
    \def\br{\hspace*{\fill} \\* }
    % Math Jax compatability definitions
    \def\gt{>}
    \def\lt{<}
    % Document parameters
    \title{Project-3-RL}
    
    
    

    % Pygments definitions
    
\makeatletter
\def\PY@reset{\let\PY@it=\relax \let\PY@bf=\relax%
    \let\PY@ul=\relax \let\PY@tc=\relax%
    \let\PY@bc=\relax \let\PY@ff=\relax}
\def\PY@tok#1{\csname PY@tok@#1\endcsname}
\def\PY@toks#1+{\ifx\relax#1\empty\else%
    \PY@tok{#1}\expandafter\PY@toks\fi}
\def\PY@do#1{\PY@bc{\PY@tc{\PY@ul{%
    \PY@it{\PY@bf{\PY@ff{#1}}}}}}}
\def\PY#1#2{\PY@reset\PY@toks#1+\relax+\PY@do{#2}}

\expandafter\def\csname PY@tok@w\endcsname{\def\PY@tc##1{\textcolor[rgb]{0.73,0.73,0.73}{##1}}}
\expandafter\def\csname PY@tok@c\endcsname{\let\PY@it=\textit\def\PY@tc##1{\textcolor[rgb]{0.25,0.50,0.50}{##1}}}
\expandafter\def\csname PY@tok@cp\endcsname{\def\PY@tc##1{\textcolor[rgb]{0.74,0.48,0.00}{##1}}}
\expandafter\def\csname PY@tok@k\endcsname{\let\PY@bf=\textbf\def\PY@tc##1{\textcolor[rgb]{0.00,0.50,0.00}{##1}}}
\expandafter\def\csname PY@tok@kp\endcsname{\def\PY@tc##1{\textcolor[rgb]{0.00,0.50,0.00}{##1}}}
\expandafter\def\csname PY@tok@kt\endcsname{\def\PY@tc##1{\textcolor[rgb]{0.69,0.00,0.25}{##1}}}
\expandafter\def\csname PY@tok@o\endcsname{\def\PY@tc##1{\textcolor[rgb]{0.40,0.40,0.40}{##1}}}
\expandafter\def\csname PY@tok@ow\endcsname{\let\PY@bf=\textbf\def\PY@tc##1{\textcolor[rgb]{0.67,0.13,1.00}{##1}}}
\expandafter\def\csname PY@tok@nb\endcsname{\def\PY@tc##1{\textcolor[rgb]{0.00,0.50,0.00}{##1}}}
\expandafter\def\csname PY@tok@nf\endcsname{\def\PY@tc##1{\textcolor[rgb]{0.00,0.00,1.00}{##1}}}
\expandafter\def\csname PY@tok@nc\endcsname{\let\PY@bf=\textbf\def\PY@tc##1{\textcolor[rgb]{0.00,0.00,1.00}{##1}}}
\expandafter\def\csname PY@tok@nn\endcsname{\let\PY@bf=\textbf\def\PY@tc##1{\textcolor[rgb]{0.00,0.00,1.00}{##1}}}
\expandafter\def\csname PY@tok@ne\endcsname{\let\PY@bf=\textbf\def\PY@tc##1{\textcolor[rgb]{0.82,0.25,0.23}{##1}}}
\expandafter\def\csname PY@tok@nv\endcsname{\def\PY@tc##1{\textcolor[rgb]{0.10,0.09,0.49}{##1}}}
\expandafter\def\csname PY@tok@no\endcsname{\def\PY@tc##1{\textcolor[rgb]{0.53,0.00,0.00}{##1}}}
\expandafter\def\csname PY@tok@nl\endcsname{\def\PY@tc##1{\textcolor[rgb]{0.63,0.63,0.00}{##1}}}
\expandafter\def\csname PY@tok@ni\endcsname{\let\PY@bf=\textbf\def\PY@tc##1{\textcolor[rgb]{0.60,0.60,0.60}{##1}}}
\expandafter\def\csname PY@tok@na\endcsname{\def\PY@tc##1{\textcolor[rgb]{0.49,0.56,0.16}{##1}}}
\expandafter\def\csname PY@tok@nt\endcsname{\let\PY@bf=\textbf\def\PY@tc##1{\textcolor[rgb]{0.00,0.50,0.00}{##1}}}
\expandafter\def\csname PY@tok@nd\endcsname{\def\PY@tc##1{\textcolor[rgb]{0.67,0.13,1.00}{##1}}}
\expandafter\def\csname PY@tok@s\endcsname{\def\PY@tc##1{\textcolor[rgb]{0.73,0.13,0.13}{##1}}}
\expandafter\def\csname PY@tok@sd\endcsname{\let\PY@it=\textit\def\PY@tc##1{\textcolor[rgb]{0.73,0.13,0.13}{##1}}}
\expandafter\def\csname PY@tok@si\endcsname{\let\PY@bf=\textbf\def\PY@tc##1{\textcolor[rgb]{0.73,0.40,0.53}{##1}}}
\expandafter\def\csname PY@tok@se\endcsname{\let\PY@bf=\textbf\def\PY@tc##1{\textcolor[rgb]{0.73,0.40,0.13}{##1}}}
\expandafter\def\csname PY@tok@sr\endcsname{\def\PY@tc##1{\textcolor[rgb]{0.73,0.40,0.53}{##1}}}
\expandafter\def\csname PY@tok@ss\endcsname{\def\PY@tc##1{\textcolor[rgb]{0.10,0.09,0.49}{##1}}}
\expandafter\def\csname PY@tok@sx\endcsname{\def\PY@tc##1{\textcolor[rgb]{0.00,0.50,0.00}{##1}}}
\expandafter\def\csname PY@tok@m\endcsname{\def\PY@tc##1{\textcolor[rgb]{0.40,0.40,0.40}{##1}}}
\expandafter\def\csname PY@tok@gh\endcsname{\let\PY@bf=\textbf\def\PY@tc##1{\textcolor[rgb]{0.00,0.00,0.50}{##1}}}
\expandafter\def\csname PY@tok@gu\endcsname{\let\PY@bf=\textbf\def\PY@tc##1{\textcolor[rgb]{0.50,0.00,0.50}{##1}}}
\expandafter\def\csname PY@tok@gd\endcsname{\def\PY@tc##1{\textcolor[rgb]{0.63,0.00,0.00}{##1}}}
\expandafter\def\csname PY@tok@gi\endcsname{\def\PY@tc##1{\textcolor[rgb]{0.00,0.63,0.00}{##1}}}
\expandafter\def\csname PY@tok@gr\endcsname{\def\PY@tc##1{\textcolor[rgb]{1.00,0.00,0.00}{##1}}}
\expandafter\def\csname PY@tok@ge\endcsname{\let\PY@it=\textit}
\expandafter\def\csname PY@tok@gs\endcsname{\let\PY@bf=\textbf}
\expandafter\def\csname PY@tok@gp\endcsname{\let\PY@bf=\textbf\def\PY@tc##1{\textcolor[rgb]{0.00,0.00,0.50}{##1}}}
\expandafter\def\csname PY@tok@go\endcsname{\def\PY@tc##1{\textcolor[rgb]{0.53,0.53,0.53}{##1}}}
\expandafter\def\csname PY@tok@gt\endcsname{\def\PY@tc##1{\textcolor[rgb]{0.00,0.27,0.87}{##1}}}
\expandafter\def\csname PY@tok@err\endcsname{\def\PY@bc##1{\setlength{\fboxsep}{0pt}\fcolorbox[rgb]{1.00,0.00,0.00}{1,1,1}{\strut ##1}}}
\expandafter\def\csname PY@tok@kc\endcsname{\let\PY@bf=\textbf\def\PY@tc##1{\textcolor[rgb]{0.00,0.50,0.00}{##1}}}
\expandafter\def\csname PY@tok@kd\endcsname{\let\PY@bf=\textbf\def\PY@tc##1{\textcolor[rgb]{0.00,0.50,0.00}{##1}}}
\expandafter\def\csname PY@tok@kn\endcsname{\let\PY@bf=\textbf\def\PY@tc##1{\textcolor[rgb]{0.00,0.50,0.00}{##1}}}
\expandafter\def\csname PY@tok@kr\endcsname{\let\PY@bf=\textbf\def\PY@tc##1{\textcolor[rgb]{0.00,0.50,0.00}{##1}}}
\expandafter\def\csname PY@tok@bp\endcsname{\def\PY@tc##1{\textcolor[rgb]{0.00,0.50,0.00}{##1}}}
\expandafter\def\csname PY@tok@fm\endcsname{\def\PY@tc##1{\textcolor[rgb]{0.00,0.00,1.00}{##1}}}
\expandafter\def\csname PY@tok@vc\endcsname{\def\PY@tc##1{\textcolor[rgb]{0.10,0.09,0.49}{##1}}}
\expandafter\def\csname PY@tok@vg\endcsname{\def\PY@tc##1{\textcolor[rgb]{0.10,0.09,0.49}{##1}}}
\expandafter\def\csname PY@tok@vi\endcsname{\def\PY@tc##1{\textcolor[rgb]{0.10,0.09,0.49}{##1}}}
\expandafter\def\csname PY@tok@vm\endcsname{\def\PY@tc##1{\textcolor[rgb]{0.10,0.09,0.49}{##1}}}
\expandafter\def\csname PY@tok@sa\endcsname{\def\PY@tc##1{\textcolor[rgb]{0.73,0.13,0.13}{##1}}}
\expandafter\def\csname PY@tok@sb\endcsname{\def\PY@tc##1{\textcolor[rgb]{0.73,0.13,0.13}{##1}}}
\expandafter\def\csname PY@tok@sc\endcsname{\def\PY@tc##1{\textcolor[rgb]{0.73,0.13,0.13}{##1}}}
\expandafter\def\csname PY@tok@dl\endcsname{\def\PY@tc##1{\textcolor[rgb]{0.73,0.13,0.13}{##1}}}
\expandafter\def\csname PY@tok@s2\endcsname{\def\PY@tc##1{\textcolor[rgb]{0.73,0.13,0.13}{##1}}}
\expandafter\def\csname PY@tok@sh\endcsname{\def\PY@tc##1{\textcolor[rgb]{0.73,0.13,0.13}{##1}}}
\expandafter\def\csname PY@tok@s1\endcsname{\def\PY@tc##1{\textcolor[rgb]{0.73,0.13,0.13}{##1}}}
\expandafter\def\csname PY@tok@mb\endcsname{\def\PY@tc##1{\textcolor[rgb]{0.40,0.40,0.40}{##1}}}
\expandafter\def\csname PY@tok@mf\endcsname{\def\PY@tc##1{\textcolor[rgb]{0.40,0.40,0.40}{##1}}}
\expandafter\def\csname PY@tok@mh\endcsname{\def\PY@tc##1{\textcolor[rgb]{0.40,0.40,0.40}{##1}}}
\expandafter\def\csname PY@tok@mi\endcsname{\def\PY@tc##1{\textcolor[rgb]{0.40,0.40,0.40}{##1}}}
\expandafter\def\csname PY@tok@il\endcsname{\def\PY@tc##1{\textcolor[rgb]{0.40,0.40,0.40}{##1}}}
\expandafter\def\csname PY@tok@mo\endcsname{\def\PY@tc##1{\textcolor[rgb]{0.40,0.40,0.40}{##1}}}
\expandafter\def\csname PY@tok@ch\endcsname{\let\PY@it=\textit\def\PY@tc##1{\textcolor[rgb]{0.25,0.50,0.50}{##1}}}
\expandafter\def\csname PY@tok@cm\endcsname{\let\PY@it=\textit\def\PY@tc##1{\textcolor[rgb]{0.25,0.50,0.50}{##1}}}
\expandafter\def\csname PY@tok@cpf\endcsname{\let\PY@it=\textit\def\PY@tc##1{\textcolor[rgb]{0.25,0.50,0.50}{##1}}}
\expandafter\def\csname PY@tok@c1\endcsname{\let\PY@it=\textit\def\PY@tc##1{\textcolor[rgb]{0.25,0.50,0.50}{##1}}}
\expandafter\def\csname PY@tok@cs\endcsname{\let\PY@it=\textit\def\PY@tc##1{\textcolor[rgb]{0.25,0.50,0.50}{##1}}}

\def\PYZbs{\char`\\}
\def\PYZus{\char`\_}
\def\PYZob{\char`\{}
\def\PYZcb{\char`\}}
\def\PYZca{\char`\^}
\def\PYZam{\char`\&}
\def\PYZlt{\char`\<}
\def\PYZgt{\char`\>}
\def\PYZsh{\char`\#}
\def\PYZpc{\char`\%}
\def\PYZdl{\char`\$}
\def\PYZhy{\char`\-}
\def\PYZsq{\char`\'}
\def\PYZdq{\char`\"}
\def\PYZti{\char`\~}
% for compatibility with earlier versions
\def\PYZat{@}
\def\PYZlb{[}
\def\PYZrb{]}
\makeatother


    % Exact colors from NB
    \definecolor{incolor}{rgb}{0.0, 0.0, 0.5}
    \definecolor{outcolor}{rgb}{0.545, 0.0, 0.0}



    
    % Prevent overflowing lines due to hard-to-break entities
    \sloppy 
    % Setup hyperref package
    \hypersetup{
      breaklinks=true,  % so long urls are correctly broken across lines
      colorlinks=true,
      urlcolor=urlcolor,
      linkcolor=linkcolor,
      citecolor=citecolor,
      }
    % Slightly bigger margins than the latex defaults
    
    \geometry{verbose,tmargin=1in,bmargin=1in,lmargin=1in,rmargin=1in}
    
    

    \begin{document}
    
    
    \maketitle
    
    

    
    \section{Reinforcement Learning}\label{reinforcement-learning}

\subsubsection{For the second part of this project we will be
implementing a simple Q-Learning algorithm on an RL environment called
Cart Pole. The idea of Q-Learning is to try to estimate the expected
future reward or Q-value of taking a certain action. Then at any given
step we take the action with the most expected future
reward.}\label{for-the-second-part-of-this-project-we-will-be-implementing-a-simple-q-learning-algorithm-on-an-rl-environment-called-cart-pole.-the-idea-of-q-learning-is-to-try-to-estimate-the-expected-future-reward-or-q-value-of-taking-a-certain-action.-then-at-any-given-step-we-take-the-action-with-the-most-expected-future-reward.}

\subsubsection{In reinforcement learning, we refer to algorithms that
attempt to solve environments as "agents", so in this part of the
project we will be making a Deep Q Network Agent that will solve the
Cart Pole
environment.}\label{in-reinforcement-learning-we-refer-to-algorithms-that-attempt-to-solve-environments-as-agents-so-in-this-part-of-the-project-we-will-be-making-a-deep-q-network-agent-that-will-solve-the-cart-pole-environment.}

    \begin{Verbatim}[commandchars=\\\{\}]
{\color{incolor}In [{\color{incolor}1}]:} \PY{o}{!}pip install gym tqdm
\end{Verbatim}


    \begin{Verbatim}[commandchars=\\\{\}]
Collecting gym
  Downloading https://files.pythonhosted.org/packages/9b/50/ed4a03d2be47ffd043be2ee514f329ce45d98a30fe2d1b9c61dea5a9d861/gym-0.10.5.tar.gz (1.5MB)
    100\% |████████████████████████████████| 1.5MB 566kB/s 
Collecting tqdm
  Downloading https://files.pythonhosted.org/packages/78/bc/de067ab2d700b91717dc5459d86a1877e2df31abfb90ab01a5a5a5ce30b4/tqdm-4.23.0-py2.py3-none-any.whl (42kB)
    100\% |████████████████████████████████| 51kB 1.6MB/s 
Requirement already satisfied: numpy>=1.10.4 in /anaconda3/envs/py36/lib/python3.6/site-packages (from gym)
Requirement already satisfied: requests>=2.0 in /anaconda3/envs/py36/lib/python3.6/site-packages (from gym)
Requirement already satisfied: six in /anaconda3/envs/py36/lib/python3.6/site-packages (from gym)
Collecting pyglet>=1.2.0 (from gym)
  Downloading https://files.pythonhosted.org/packages/1c/fc/dad5eaaab68f0c21e2f906a94ddb98175662cc5a654eee404d59554ce0fa/pyglet-1.3.2-py2.py3-none-any.whl (1.0MB)
    100\% |████████████████████████████████| 1.0MB 827kB/s 
Requirement already satisfied: chardet<3.1.0,>=3.0.2 in /anaconda3/envs/py36/lib/python3.6/site-packages (from requests>=2.0->gym)
Requirement already satisfied: idna<2.7,>=2.5 in /anaconda3/envs/py36/lib/python3.6/site-packages (from requests>=2.0->gym)
Requirement already satisfied: urllib3<1.23,>=1.21.1 in /anaconda3/envs/py36/lib/python3.6/site-packages (from requests>=2.0->gym)
Requirement already satisfied: certifi>=2017.4.17 in /anaconda3/envs/py36/lib/python3.6/site-packages (from requests>=2.0->gym)
Collecting future (from pyglet>=1.2.0->gym)
  Downloading https://files.pythonhosted.org/packages/00/2b/8d082ddfed935f3608cc61140df6dcbf0edea1bc3ab52fb6c29ae3e81e85/future-0.16.0.tar.gz (824kB)
    100\% |████████████████████████████████| 829kB 1.0MB/s 
Building wheels for collected packages: gym, future
  Running setup.py bdist\_wheel for gym {\ldots} done
  Stored in directory: /Users/chapatel/Library/Caches/pip/wheels/cb/14/71/f4ab006b1e6ff75c2b54985c2f98d0644fffe9c1dddc670925
  Running setup.py bdist\_wheel for future {\ldots} done
  Stored in directory: /Users/chapatel/Library/Caches/pip/wheels/bf/c9/a3/c538d90ef17cf7823fa51fc701a7a7a910a80f6a405bf15b1a
Successfully built gym future
Installing collected packages: future, pyglet, gym, tqdm
Successfully installed future-0.16.0 gym-0.10.5 pyglet-1.3.2 tqdm-4.23.0
\textcolor{ansi-yellow}{You are using pip version 9.0.1, however version 10.0.1 is available.
You should consider upgrading via the 'pip install --upgrade pip' command.}

    \end{Verbatim}

    \section{Part 1: Setup the
Environment}\label{part-1-setup-the-environment}

    \begin{Verbatim}[commandchars=\\\{\}]
{\color{incolor}In [{\color{incolor}2}]:} \PY{k+kn}{import} \PY{n+nn}{gym}
        \PY{n}{env} \PY{o}{=} \PY{n}{gym}\PY{o}{.}\PY{n}{make}\PY{p}{(}\PY{l+s+s1}{\PYZsq{}}\PY{l+s+s1}{CartPole\PYZhy{}v0}\PY{l+s+s1}{\PYZsq{}}\PY{p}{)}
\end{Verbatim}


    \begin{Verbatim}[commandchars=\\\{\}]
\textcolor{ansi-yellow}{WARN: gym.spaces.Box autodetected dtype as <class 'numpy.float32'>. Please provide explicit dtype.}

    \end{Verbatim}

    \section{Part 2: Create The DQN
Agent}\label{part-2-create-the-dqn-agent}

    \begin{Verbatim}[commandchars=\\\{\}]
{\color{incolor}In [{\color{incolor}3}]:} \PY{k+kn}{import} \PY{n+nn}{keras} 
        \PY{k+kn}{from} \PY{n+nn}{keras}\PY{n+nn}{.}\PY{n+nn}{models} \PY{k}{import} \PY{n}{Sequential}
        \PY{k+kn}{from} \PY{n+nn}{keras}\PY{n+nn}{.}\PY{n+nn}{layers} \PY{k}{import} \PY{n}{Dense}
        \PY{k+kn}{from} \PY{n+nn}{keras}\PY{n+nn}{.}\PY{n+nn}{layers} \PY{k}{import} \PY{n}{Activation}
        \PY{k+kn}{from} \PY{n+nn}{collections} \PY{k}{import} \PY{n}{deque}
        \PY{k+kn}{import} \PY{n+nn}{random}
        \PY{k+kn}{from} \PY{n+nn}{keras}\PY{n+nn}{.}\PY{n+nn}{optimizers} \PY{k}{import} \PY{n}{Adam}
        
        \PY{k+kn}{import} \PY{n+nn}{numpy} \PY{k}{as} \PY{n+nn}{np}
        
        
        \PY{k}{class} \PY{n+nc}{DQNAgent}\PY{p}{:}
            
            \PY{k}{def} \PY{n+nf}{\PYZus{}\PYZus{}init\PYZus{}\PYZus{}}\PY{p}{(}\PY{n+nb+bp}{self}\PY{p}{,} \PY{n}{env}\PY{p}{,} \PY{n}{replay\PYZus{}size}\PY{o}{=}\PY{l+m+mi}{1000}\PY{p}{,} \PY{n}{epsilon}\PY{o}{=}\PY{l+m+mf}{1.0}\PY{p}{,} \PY{n}{epsilon\PYZus{}min}\PY{o}{=}\PY{l+m+mf}{0.01}\PY{p}{,} \PY{n}{epsilon\PYZus{}decay}\PY{o}{=}\PY{l+m+mf}{0.995}\PY{p}{,} \PY{n}{gamma}\PY{o}{=}\PY{l+m+mf}{0.99}\PY{p}{)}\PY{p}{:}
                \PY{n+nb+bp}{self}\PY{o}{.}\PY{n}{state\PYZus{}size} \PY{o}{=} \PY{n}{env}\PY{o}{.}\PY{n}{observation\PYZus{}space}\PY{o}{.}\PY{n}{shape}\PY{p}{[}\PY{l+m+mi}{0}\PY{p}{]}
                \PY{n+nb+bp}{self}\PY{o}{.}\PY{n}{num\PYZus{}actions} \PY{o}{=} \PY{n}{env}\PY{o}{.}\PY{n}{action\PYZus{}space}\PY{o}{.}\PY{n}{n}
                \PY{n+nb+bp}{self}\PY{o}{.}\PY{n}{model} \PY{o}{=} \PY{n+nb+bp}{self}\PY{o}{.}\PY{n}{build\PYZus{}model}\PY{p}{(}\PY{p}{)}
                \PY{n+nb+bp}{self}\PY{o}{.}\PY{n}{replay\PYZus{}buffer} \PY{o}{=} \PY{n}{deque}\PY{p}{(}\PY{n}{maxlen}\PY{o}{=}\PY{n}{replay\PYZus{}size}\PY{p}{)}
                \PY{n+nb+bp}{self}\PY{o}{.}\PY{n}{epsilon} \PY{o}{=} \PY{n}{epsilon}
                \PY{n+nb+bp}{self}\PY{o}{.}\PY{n}{epsilon\PYZus{}min} \PY{o}{=} \PY{n}{epsilon\PYZus{}min}
                \PY{n+nb+bp}{self}\PY{o}{.}\PY{n}{epsilon\PYZus{}decay} \PY{o}{=} \PY{n}{epsilon\PYZus{}decay}
                \PY{n+nb+bp}{self}\PY{o}{.}\PY{n}{gamma} \PY{o}{=} \PY{n}{gamma}
        
                
            \PY{k}{def} \PY{n+nf}{build\PYZus{}model}\PY{p}{(}\PY{n+nb+bp}{self}\PY{p}{)}\PY{p}{:}
                \PY{n}{model} \PY{o}{=} \PY{n}{Sequential}\PY{p}{(}\PY{p}{)}
                \PY{c+c1}{\PYZsh{} TODO: add 2 dense layers each with 32 neurons, the input dim to the first}
                \PY{c+c1}{\PYZsh{} layer should be the state size, also add relu activations, for both these layers}
                \PY{c+c1}{\PYZsh{} Then add another Dense layer with num\PYZus{}actions neurons.}
                \PY{c+c1}{\PYZsh{} Then use model.compile to compile the model with mse loss and an Adam optimizer}
                \PY{c+c1}{\PYZsh{} with learning rate 0.001.}
                \PY{n}{model}\PY{o}{.}\PY{n}{add}\PY{p}{(}\PY{n}{Dense}\PY{p}{(}\PY{l+m+mi}{32}\PY{p}{,} \PY{n}{input\PYZus{}dim} \PY{o}{=} \PY{n+nb+bp}{self}\PY{o}{.}\PY{n}{state\PYZus{}size}\PY{p}{)}\PY{p}{)}
                \PY{n}{model}\PY{o}{.}\PY{n}{add}\PY{p}{(}\PY{n}{Activation}\PY{p}{(}\PY{l+s+s2}{\PYZdq{}}\PY{l+s+s2}{relu}\PY{l+s+s2}{\PYZdq{}}\PY{p}{)}\PY{p}{)}
                \PY{n}{model}\PY{o}{.}\PY{n}{add}\PY{p}{(}\PY{n}{Dense}\PY{p}{(}\PY{l+m+mi}{32}\PY{p}{)}\PY{p}{)}
                \PY{n}{model}\PY{o}{.}\PY{n}{add}\PY{p}{(}\PY{n}{Activation}\PY{p}{(}\PY{l+s+s2}{\PYZdq{}}\PY{l+s+s2}{relu}\PY{l+s+s2}{\PYZdq{}}\PY{p}{)}\PY{p}{)}
                \PY{n}{model}\PY{o}{.}\PY{n}{add}\PY{p}{(}\PY{n}{Dense}\PY{p}{(}\PY{n+nb+bp}{self}\PY{o}{.}\PY{n}{num\PYZus{}actions}\PY{p}{)}\PY{p}{)}
                \PY{n}{keras}\PY{o}{.}\PY{n}{optimizers}\PY{o}{.}\PY{n}{Adam}\PY{p}{(}\PY{n}{lr}\PY{o}{=}\PY{l+m+mf}{0.001}\PY{p}{)}
                \PY{n}{model}\PY{o}{.}\PY{n}{compile}\PY{p}{(}\PY{n}{optimizer} \PY{o}{=} \PY{l+s+s2}{\PYZdq{}}\PY{l+s+s2}{Adam}\PY{l+s+s2}{\PYZdq{}}\PY{p}{,} \PY{n}{loss}\PY{o}{=}\PY{l+s+s2}{\PYZdq{}}\PY{l+s+s2}{mse}\PY{l+s+s2}{\PYZdq{}}\PY{p}{)}
        
                
                \PY{k}{return} \PY{n}{model}
                
            \PY{k}{def} \PY{n+nf}{action}\PY{p}{(}\PY{n+nb+bp}{self}\PY{p}{,} \PY{n}{state}\PY{p}{)}\PY{p}{:}
                \PY{c+c1}{\PYZsh{} Whenever a random number between 0 and 1 is less than epsilon we want to return}
                \PY{c+c1}{\PYZsh{} a random action. This means that with probability epsilon we return a random action.}
                \PY{k}{if} \PY{n}{np}\PY{o}{.}\PY{n}{random}\PY{o}{.}\PY{n}{random}\PY{p}{(}\PY{p}{)} \PY{o}{\PYZlt{}}\PY{o}{=} \PY{n+nb+bp}{self}\PY{o}{.}\PY{n}{epsilon}\PY{p}{:}
                    \PY{k}{return} \PY{n}{np}\PY{o}{.}\PY{n}{random}\PY{o}{.}\PY{n}{randint}\PY{p}{(}\PY{n+nb+bp}{self}\PY{o}{.}\PY{n}{num\PYZus{}actions}\PY{p}{)}
                    \PY{c+c1}{\PYZsh{}TODO: return random action here}
                \PY{c+c1}{\PYZsh{} Now we want to use our model to get the q values}
                \PY{c+c1}{\PYZsh{} HINT: we want to do prediction}
                
                \PY{n}{q\PYZus{}values} \PY{o}{=} \PY{n+nb+bp}{self}\PY{o}{.}\PY{n}{model}\PY{o}{.}\PY{n}{predict}\PY{p}{(}\PY{n}{state}\PY{p}{)}
                \PY{k}{return} \PY{n}{np}\PY{o}{.}\PY{n}{argmax}\PY{p}{(}\PY{n}{q\PYZus{}values}\PY{p}{[}\PY{l+m+mi}{0}\PY{p}{]}\PY{p}{)}
            
            \PY{k}{def} \PY{n+nf}{add\PYZus{}to\PYZus{}replay\PYZus{}buffer}\PY{p}{(}\PY{n+nb+bp}{self}\PY{p}{,} \PY{n}{state}\PY{p}{,} \PY{n}{action}\PY{p}{,} \PY{n}{reward}\PY{p}{,} \PY{n}{next\PYZus{}state}\PY{p}{,} \PY{n}{done}\PY{p}{)}\PY{p}{:}
                \PY{n+nb+bp}{self}\PY{o}{.}\PY{n}{replay\PYZus{}buffer}\PY{o}{.}\PY{n}{append}\PY{p}{(}\PY{p}{(}\PY{n}{state}\PY{p}{,} \PY{n}{action}\PY{p}{,} \PY{n}{reward}\PY{p}{,} \PY{n}{next\PYZus{}state}\PY{p}{,} \PY{n}{done}\PY{p}{)}\PY{p}{)}
        
            \PY{k}{def} \PY{n+nf}{train\PYZus{}batch\PYZus{}from\PYZus{}replay}\PY{p}{(}\PY{n+nb+bp}{self}\PY{p}{,} \PY{n}{batch\PYZus{}size}\PY{p}{)}\PY{p}{:}
                \PY{c+c1}{\PYZsh{} if we don\PYZsq{}t have enough samples in our replay buffer just return}
                \PY{k}{if} \PY{n+nb}{len}\PY{p}{(}\PY{n+nb+bp}{self}\PY{o}{.}\PY{n}{replay\PYZus{}buffer}\PY{p}{)} \PY{o}{\PYZlt{}} \PY{n}{batch\PYZus{}size}\PY{p}{:}
                    \PY{k}{return} \PY{k+kc}{False}
                \PY{c+c1}{\PYZsh{} TODO: randomly sample batch\PYZus{}size samples from the replay buffer}
                \PY{c+c1}{\PYZsh{} hint: use random.sample}
                \PY{n}{minibatch} \PY{o}{=} \PY{n}{random}\PY{o}{.}\PY{n}{sample}\PY{p}{(}\PY{n+nb+bp}{self}\PY{o}{.}\PY{n}{replay\PYZus{}buffer}\PY{p}{,} \PY{n}{batch\PYZus{}size}\PY{p}{)}
                \PY{k}{for} \PY{n}{state}\PY{p}{,} \PY{n}{action}\PY{p}{,} \PY{n}{reward}\PY{p}{,} \PY{n}{next\PYZus{}state}\PY{p}{,} \PY{n}{done} \PY{o+ow}{in} \PY{n}{minibatch}\PY{p}{:}
                    \PY{n}{target} \PY{o}{=} \PY{n}{reward}
                    \PY{k}{if} \PY{o+ow}{not} \PY{n}{done}\PY{p}{:}
                        \PY{n}{next\PYZus{}Qs} \PY{o}{=} \PY{n+nb+bp}{self}\PY{o}{.}\PY{n}{model}\PY{o}{.}\PY{n}{predict}\PY{p}{(}\PY{n}{next\PYZus{}state}\PY{p}{)}\PY{p}{[}\PY{l+m+mi}{0}\PY{p}{]}
                        \PY{c+c1}{\PYZsh{} TODO: we want to add to our target GAMMA * max Q(next\PYZus{}state)}
                        \PY{n}{target} \PY{o}{+}\PY{o}{=} \PY{n+nb+bp}{self}\PY{o}{.}\PY{n}{gamma} \PY{o}{*} \PY{n}{np}\PY{o}{.}\PY{n}{max}\PY{p}{(}\PY{n}{next\PYZus{}Qs}\PY{p}{)}
        
                    \PY{c+c1}{\PYZsh{} our target should only take into account the current action}
                    \PY{c+c1}{\PYZsh{} so we set all the Q values except the current action, to the }
                    \PY{c+c1}{\PYZsh{} current output of our model so that they get ignored in the loss function.}
                    \PY{n}{target\PYZus{}Qs} \PY{o}{=} \PY{n+nb+bp}{self}\PY{o}{.}\PY{n}{model}\PY{o}{.}\PY{n}{predict}\PY{p}{(}\PY{n}{state}\PY{p}{)}
                    \PY{n}{target\PYZus{}Qs}\PY{p}{[}\PY{l+m+mi}{0}\PY{p}{]}\PY{p}{[}\PY{n}{action}\PY{p}{]} \PY{o}{=} \PY{n}{target}
                    \PY{n+nb+bp}{self}\PY{o}{.}\PY{n}{model}\PY{o}{.}\PY{n}{fit}\PY{p}{(}\PY{n}{state}\PY{p}{,} \PY{n}{target\PYZus{}Qs}\PY{p}{,} \PY{n}{epochs}\PY{o}{=}\PY{l+m+mi}{1}\PY{p}{,} \PY{n}{verbose}\PY{o}{=}\PY{l+m+mi}{0}\PY{p}{)}
                
                \PY{c+c1}{\PYZsh{} Now we want to slowly decay how many random actions we take}
                \PY{c+c1}{\PYZsh{} to do this we can multiply epsilon by our epsilon decay parameter}
                \PY{c+c1}{\PYZsh{} each iteration}
                \PY{k}{if} \PY{n+nb+bp}{self}\PY{o}{.}\PY{n}{epsilon} \PY{o}{\PYZgt{}} \PY{n+nb+bp}{self}\PY{o}{.}\PY{n}{epsilon\PYZus{}min}\PY{p}{:}
                    \PY{n+nb+bp}{self}\PY{o}{.}\PY{n}{epsilon} \PY{o}{*}\PY{o}{=} \PY{n+nb+bp}{self}\PY{o}{.}\PY{n}{epsilon\PYZus{}decay}
\end{Verbatim}


    \begin{Verbatim}[commandchars=\\\{\}]
Using TensorFlow backend.

    \end{Verbatim}

    \section{Part 3: Train the Model}\label{part-3-train-the-model}

    \begin{Verbatim}[commandchars=\\\{\}]
{\color{incolor}In [{\color{incolor}4}]:} \PY{n}{agent} \PY{o}{=} \PY{n}{DQNAgent}\PY{p}{(}\PY{n}{env}\PY{p}{)}
\end{Verbatim}


    \begin{Verbatim}[commandchars=\\\{\}]
{\color{incolor}In [{\color{incolor}5}]:} \PY{k+kn}{from} \PY{n+nn}{tqdm} \PY{k}{import} \PY{n}{tqdm}
        
        \PY{n}{done} \PY{o}{=} \PY{k+kc}{False}
        \PY{n}{batch\PYZus{}size} \PY{o}{=} \PY{l+m+mi}{32}
        \PY{n}{num\PYZus{}episodes} \PY{o}{=} \PY{l+m+mi}{800}
        
        \PY{k}{for} \PY{n}{episode} \PY{o+ow}{in} \PY{n}{tqdm}\PY{p}{(}\PY{n+nb}{range}\PY{p}{(}\PY{n}{num\PYZus{}episodes}\PY{p}{)}\PY{p}{)}\PY{p}{:}
            \PY{n}{state} \PY{o}{=} \PY{n}{env}\PY{o}{.}\PY{n}{reset}\PY{p}{(}\PY{p}{)}
            \PY{n}{state} \PY{o}{=} \PY{n}{np}\PY{o}{.}\PY{n}{reshape}\PY{p}{(}\PY{n}{state}\PY{p}{,} \PY{p}{[}\PY{l+m+mi}{1}\PY{p}{,} \PY{n}{agent}\PY{o}{.}\PY{n}{state\PYZus{}size}\PY{p}{]}\PY{p}{)}
            
            \PY{k}{for} \PY{n}{t} \PY{o+ow}{in} \PY{n+nb}{range}\PY{p}{(}\PY{l+m+mi}{200}\PY{p}{)}\PY{p}{:}
                \PY{n}{action} \PY{o}{=} \PY{n}{agent}\PY{o}{.}\PY{n}{action}\PY{p}{(}\PY{n}{state}\PY{p}{)}
                \PY{n}{next\PYZus{}state}\PY{p}{,} \PY{n}{reward}\PY{p}{,} \PY{n}{done}\PY{p}{,} \PY{n}{\PYZus{}} \PY{o}{=} \PY{n}{env}\PY{o}{.}\PY{n}{step}\PY{p}{(}\PY{n}{action}\PY{p}{)}
                \PY{n}{reward} \PY{o}{=} \PY{n}{reward} \PY{k}{if} \PY{o+ow}{not} \PY{n}{done} \PY{k}{else} \PY{l+m+mi}{100}
                \PY{n}{next\PYZus{}state} \PY{o}{=} \PY{n}{np}\PY{o}{.}\PY{n}{reshape}\PY{p}{(}\PY{n}{next\PYZus{}state}\PY{p}{,} \PY{p}{[}\PY{l+m+mi}{1}\PY{p}{,} \PY{n}{agent}\PY{o}{.}\PY{n}{state\PYZus{}size}\PY{p}{]}\PY{p}{)}
                \PY{n}{agent}\PY{o}{.}\PY{n}{add\PYZus{}to\PYZus{}replay\PYZus{}buffer}\PY{p}{(}\PY{n}{state}\PY{p}{,} \PY{n}{action}\PY{p}{,} \PY{n}{reward}\PY{p}{,} \PY{n}{next\PYZus{}state}\PY{p}{,} \PY{n}{done}\PY{p}{)}
                \PY{c+c1}{\PYZsh{} TODO: add this sample to the replay buffer}
                
                \PY{n}{state} \PY{o}{=} \PY{n}{next\PYZus{}state}
                
                \PY{c+c1}{\PYZsh{} TODO: train on a batch from the replay buffer}
                \PY{n}{agent}\PY{o}{.}\PY{n}{train\PYZus{}batch\PYZus{}from\PYZus{}replay}\PY{p}{(}\PY{n}{batch\PYZus{}size}\PY{p}{)}
                \PY{k}{if} \PY{n}{done}\PY{p}{:} 
                    \PY{k}{break}
\end{Verbatim}


    \begin{Verbatim}[commandchars=\\\{\}]
100\%|██████████| 800/800 [34:59<00:00,  2.62s/it]

    \end{Verbatim}

    \section{Part 4: Test the Model}\label{part-4-test-the-model}

    \begin{Verbatim}[commandchars=\\\{\}]
{\color{incolor}In [{\color{incolor}6}]:} \PY{c+c1}{\PYZsh{}TODO: set the agent\PYZsq{}s epsilon so that we don\PYZsq{}t take any random actions.}
        \PY{k}{for} \PY{n}{\PYZus{}} \PY{o+ow}{in} \PY{n+nb}{range}\PY{p}{(}\PY{l+m+mi}{10}\PY{p}{)}\PY{p}{:}
            \PY{n}{state} \PY{o}{=} \PY{n}{env}\PY{o}{.}\PY{n}{reset}\PY{p}{(}\PY{p}{)}
            \PY{n}{state} \PY{o}{=} \PY{n}{np}\PY{o}{.}\PY{n}{reshape}\PY{p}{(}\PY{n}{state}\PY{p}{,} \PY{p}{[}\PY{l+m+mi}{1}\PY{p}{,} \PY{n}{agent}\PY{o}{.}\PY{n}{state\PYZus{}size}\PY{p}{]}\PY{p}{)}
            \PY{n}{agent}\PY{o}{.}\PY{n}{episilon} \PY{o}{=} \PY{o}{\PYZhy{}}\PY{l+m+mi}{1}
            \PY{n}{total\PYZus{}reward} \PY{o}{=} \PY{l+m+mi}{0}
            \PY{k}{for} \PY{n}{t} \PY{o+ow}{in} \PY{n+nb}{range}\PY{p}{(}\PY{l+m+mi}{200}\PY{p}{)}\PY{p}{:}
                \PY{n}{action} \PY{o}{=} \PY{n}{agent}\PY{o}{.}\PY{n}{action}\PY{p}{(}\PY{n}{state}\PY{p}{)}
                \PY{n}{next\PYZus{}state}\PY{p}{,} \PY{n}{reward}\PY{p}{,} \PY{n}{done}\PY{p}{,} \PY{n}{\PYZus{}} \PY{o}{=} \PY{n}{env}\PY{o}{.}\PY{n}{step}\PY{p}{(}\PY{n}{action}\PY{p}{)}
                \PY{n}{total\PYZus{}reward} \PY{o}{+}\PY{o}{=} \PY{n}{reward}
                \PY{n}{state} \PY{o}{=} \PY{n}{np}\PY{o}{.}\PY{n}{reshape}\PY{p}{(}\PY{n}{next\PYZus{}state}\PY{p}{,} \PY{p}{[}\PY{l+m+mi}{1}\PY{p}{,} \PY{n}{agent}\PY{o}{.}\PY{n}{state\PYZus{}size}\PY{p}{]}\PY{p}{)}
                \PY{c+c1}{\PYZsh{} TODO: if you want to see the rendered version of your agent running}
                \PY{c+c1}{\PYZsh{} uncomment this line}
                \PY{c+c1}{\PYZsh{}env.render()}
            \PY{n+nb}{print}\PY{p}{(}\PY{n}{total\PYZus{}reward}\PY{p}{)}
\end{Verbatim}


    \begin{Verbatim}[commandchars=\\\{\}]
\textcolor{ansi-yellow}{WARN: You are calling 'step()' even though this environment has already returned done = True. You should always call 'reset()' once you receive 'done = True' -- any further steps are undefined behavior.}
40.0
\textcolor{ansi-yellow}{WARN: You are calling 'step()' even though this environment has already returned done = True. You should always call 'reset()' once you receive 'done = True' -- any further steps are undefined behavior.}
23.0
\textcolor{ansi-yellow}{WARN: You are calling 'step()' even though this environment has already returned done = True. You should always call 'reset()' once you receive 'done = True' -- any further steps are undefined behavior.}
27.0
\textcolor{ansi-yellow}{WARN: You are calling 'step()' even though this environment has already returned done = True. You should always call 'reset()' once you receive 'done = True' -- any further steps are undefined behavior.}
21.0
\textcolor{ansi-yellow}{WARN: You are calling 'step()' even though this environment has already returned done = True. You should always call 'reset()' once you receive 'done = True' -- any further steps are undefined behavior.}
35.0
\textcolor{ansi-yellow}{WARN: You are calling 'step()' even though this environment has already returned done = True. You should always call 'reset()' once you receive 'done = True' -- any further steps are undefined behavior.}
21.0
\textcolor{ansi-yellow}{WARN: You are calling 'step()' even though this environment has already returned done = True. You should always call 'reset()' once you receive 'done = True' -- any further steps are undefined behavior.}
24.0
\textcolor{ansi-yellow}{WARN: You are calling 'step()' even though this environment has already returned done = True. You should always call 'reset()' once you receive 'done = True' -- any further steps are undefined behavior.}
46.0
\textcolor{ansi-yellow}{WARN: You are calling 'step()' even though this environment has already returned done = True. You should always call 'reset()' once you receive 'done = True' -- any further steps are undefined behavior.}
20.0
\textcolor{ansi-yellow}{WARN: You are calling 'step()' even though this environment has already returned done = True. You should always call 'reset()' once you receive 'done = True' -- any further steps are undefined behavior.}
27.0

    \end{Verbatim}

    \section{Part 5: Writeup}\label{part-5-writeup}

\paragraph{Now for the writeup portion write a paragraph of your
understanding of how Deep Q Learning
works.}\label{now-for-the-writeup-portion-write-a-paragraph-of-your-understanding-of-how-deep-q-learning-works.}

Q-learning uses a simple update rule to perform q-value iteration, which
allows us to bypass the need to keep track of values, transition
functions, and reward functions. We use Deep Q-Learning to approximate
our Q-value function with the use of a Neural Network. We choose the
neuron from out network that has the highest value and take an action
corresponding to this neuron.


    % Add a bibliography block to the postdoc
    
    
    
    \end{document}
